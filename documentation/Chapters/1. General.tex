\chapter{Senku Solver: Suite Completa de Métodos Numéricos}

\section{Introducción}

En el presente documento se desarrolla una aplicación web interactiva denominada \textbf{Senku Solver} que implementa una suite completa de métodos numéricos fundamentales, abarcando tanto la búsqueda de raíces de ecuaciones no lineales como la resolución de sistemas de ecuaciones lineales mediante métodos iterativos. La aplicación está construida utilizando tecnologías web modernas como React, TypeScript y Vite, proporcionando una interfaz de usuario intuitiva y responsiva que permite a los usuarios explorar, comparar y analizar el comportamiento de diferentes algoritmos numéricos en ambas categorías.

\section{Capítulo 1: Métodos para Encontrar Raíces}

La aplicación incluye la implementación completa de seis métodos numéricos para la búsqueda de raíces de ecuaciones no lineales:

\subsection{Métodos de Intervalo}

\subsubsection{Método de Bisección}
Implementa el algoritmo clásico de bisección que garantiza la convergencia cuando la función cambia de signo en el intervalo dado. El método utiliza la fórmula:
\[
x_m = \frac{x_i + x_s}{2}
\]
donde se selecciona el subintervalo que mantiene el cambio de signo. La convergencia está garantizada con una tasa de convergencia lineal.

\subsubsection{Método de Regla Falsa (False Position)}
Mejora el método de bisección utilizando interpolación lineal para aproximar la raíz. La fórmula utilizada es:
\[
x_r = x_i - \frac{f(x_i)(x_s - x_i)}{f(x_s) - f(x_i)}
\]
Este método generalmente converge más rápido que bisección manteniendo la robustez de los métodos de intervalo.

\subsection{Métodos de Punto Fijo}

\subsubsection{Método de Punto Fijo}
Transforma la ecuación $f(x) = 0$ en la forma $x = g(x)$ y utiliza la iteración:
\[
x_{n+1} = g(x_n)
\]
La convergencia depende de que $|g'(x)| < 1$ en el intervalo de interés. La aplicación permite al usuario especificar tanto $f(x)$ como $g(x)$ para máxima flexibilidad.

\subsection{Métodos Basados en Derivadas}

\subsubsection{Método de Newton-Raphson}
Implementa el algoritmo clásico de Newton que utiliza información de la derivada:
\[
x_{n+1} = x_n - \frac{f(x_n)}{f'(x_n)}
\]
Ofrece convergencia cuadrática cuando las condiciones iniciales son apropiadas, pero requiere el cálculo de la derivada.

\subsubsection{Método de la Secante}
Aproxima la derivada utilizando dos puntos anteriores, eliminando la necesidad de calcular derivadas analíticamente:
\[
x_{n+1} = x_n - \frac{f(x_n)(x_n - x_{n-1})}{f(x_n) - f(x_{n-1})}
\]
Proporciona una convergencia superlineal (orden $\approx 1.618$) sin requerir derivadas.

\subsubsection{Método de Raíces Múltiples}
Utiliza una modificación del método de Newton para manejar raíces múltiples donde $f(x) = f'(x) = 0$:
\[
x_{n+1} = x_n - \frac{f(x_n) \cdot f'(x_n)}{[f'(x_n)]^2 - f(x_n) \cdot f''(x_n)}
\]
Este método restaura la convergencia cuadrática en presencia de raíces múltiples.

\section{Capítulo 2: Métodos Iterativos para Sistemas Lineales}

La aplicación implementa tres métodos iterativos fundamentales para resolver sistemas de ecuaciones lineales de la forma $Ax = b$, donde $A$ es una matriz cuadrada de hasta $7 \times 7$, $x$ es el vector solución y $b$ es el vector de términos independientes.

\subsection{Fundamentos Teóricos}

Todos los métodos iterativos implementados se basan en la descomposición de la matriz $A$ en la forma:
\[
A = D + L + U
\]
donde:
\begin{itemize}
    \item $D$ es la matriz diagonal
    \item $L$ es la matriz triangular inferior estricta
    \item $U$ es la matriz triangular superior estricta
\end{itemize}

La convergencia de estos métodos está determinada por el \textbf{radio espectral} $\rho$ de la matriz de iteración correspondiente. La condición necesaria y suficiente para convergencia es $\rho < 1$.

\subsection{Método de Jacobi}

\subsubsection{Formulación}
El método de Jacobi utiliza todos los valores de la iteración anterior para calcular la nueva aproximación:
\[
x_i^{(k+1)} = \frac{1}{a_{ii}}\left(b_i - \sum_{j=1, j \neq i}^{n} a_{ij} x_j^{(k)}\right)
\]

La matriz de iteración es:
\[
T_J = D^{-1}(L + U)
\]

\subsubsection{Implementación}
La implementación incluye:
\begin{itemize}
    \item Cálculo automático del radio espectral usando círculos de Gershgorin
    \item Predicción teórica de convergencia ($\rho < 1$)
    \item Validación de elementos diagonales no nulos
    \item Seguimiento detallado de iteraciones con errores absolutos y relativos
\end{itemize}

\subsection{Método de Gauss-Seidel}

\subsubsection{Formulación}
Gauss-Seidel mejora Jacobi utilizando los valores más recientes disponibles en cada iteración:
\[
x_i^{(k+1)} = \frac{1}{a_{ii}}\left(b_i - \sum_{j=1}^{i-1} a_{ij} x_j^{(k+1)} - \sum_{j=i+1}^{n} a_{ij} x_j^{(k)}\right)
\]

La matriz de iteración es:
\[
T_{GS} = (D + L)^{-1}U
\]

\subsubsection{Ventajas Implementadas}
\begin{itemize}
    \item Convergencia generalmente más rápida que Jacobi
    \item Menor uso de memoria (actualización in-place)
    \item Análisis de dominancia diagonal automático
    \item Predicción mejorada de comportamiento de convergencia
\end{itemize}

\subsection{Método SOR (Successive Over-Relaxation)}

\subsubsection{Formulación}
SOR introduce un factor de relajación $\omega$ para acelerar la convergencia:
\[
x_i^{(k+1)} = (1-\omega)x_i^{(k)} + \frac{\omega}{a_{ii}}\left(b_i - \sum_{j=1}^{i-1} a_{ij} x_j^{(k+1)} - \sum_{j=i+1}^{n} a_{ij} x_j^{(k)}\right)
\]

La matriz de iteración es:
\[
T_{SOR} = (D - \omega L)^{-1}[(1-\omega)D + \omega U]
\]

\subsubsection{Optimización del Factor $\omega$}
La implementación incluye:
\begin{itemize}
    \item \textbf{Estimación automática}: Cálculo de $\omega$ óptimo basado en propiedades de la matriz
    \item \textbf{Validación de rango}: $0 < \omega < 2$ para garantizar convergencia
    \item \textbf{Casos especiales}: $\omega = 1$ (Gauss-Seidel), $\omega < 1$ (sub-relajación), $\omega > 1$ (sobre-relajación)
    \item \textbf{Análisis de rendimiento}: Potencial aceleración hasta $2\times$ respecto a Gauss-Seidel
\end{itemize}

\subsection{Análisis de Convergencia}

\subsubsection{Radio Espectral}
Para cada método, la aplicación calcula el radio espectral utilizando aproximaciones basadas en:
\begin{itemize}
    \item Círculos de Gershgorin para estimación inicial
    \item Análisis de dominancia diagonal estricta
    \item Predicción teórica de convergencia
\end{itemize}

\subsubsection{Condiciones de Convergencia}
\begin{itemize}
    \item \textbf{Dominancia diagonal estricta}: $|a_{ii}| > \sum_{j \neq i} |a_{ij}|$ garantiza convergencia
    \item \textbf{Matrices definidas positivas}: Convergencia asegurada para Gauss-Seidel y SOR
    \item \textbf{Validación automática}: Verificación de condiciones en tiempo real
\end{itemize}

\section{Sistema de Comparación Iterativa}

\subsection{Metodología de Análisis Comparativo}

La aplicación incluye un módulo especializado de comparación que ejecuta simultáneamente los tres métodos iterativos con los mismos parámetros y matriz objetivo.

\subsubsection{Criterios de Evaluación}
\begin{enumerate}
    \item \textbf{Convergencia}: Prioridad máxima - verificación de $\rho < 1$ y convergencia efectiva
    \item \textbf{Eficiencia Iterativa}: Número de iteraciones requeridas para alcanzar la tolerancia
    \item \textbf{Tiempo de Ejecución}: Medición precisa usando \texttt{performance.now()}
    \item \textbf{Radio Espectral}: Análisis teórico de la velocidad de convergencia
    \item \textbf{Precisión Final}: Error relativo alcanzado en la solución
\end{enumerate}

\subsubsection{Proceso de Comparación Automática}
\begin{enumerate}
    \item \textbf{Configuración Uniforme}: Misma matriz, vector, tolerancia y límite de iteraciones
    \item \textbf{Ejecución Paralela Conceptual}: Medición independiente de cada método
    \item \textbf{Análisis Estadístico}: Cálculo de métricas comparativas
    \item \textbf{Identificación del Método Óptimo}: Basado en criterios ponderados
    \item \textbf{Generación de Informe}: Exportación completa en formato CSV
\end{enumerate}

\subsection{Métricas del Informe Comparativo}

\subsubsection{Tabla Comparativa Principal}
\begin{itemize}
    \item Estado de convergencia con badges visuales
    \item Radio espectral calculado para cada método
    \item Predicción teórica de convergencia ($\rho < 1$)
    \item Número de iteraciones y tiempo de ejecución
    \item Error final y solución completa
    \item Factor $\omega$ utilizado en SOR
\end{itemize}

\subsubsection{Estadísticas Agregadas}
\begin{itemize}
    \item Número de métodos convergentes
    \item Promedio de iteraciones entre métodos exitosos
    \item Mejor radio espectral obtenido
    \item Tiempo total de análisis
\end{itemize}

\subsubsection{Informe Automático de Análisis}
El sistema genera recomendaciones automáticas basadas en:
\begin{itemize}
    \item \textbf{Análisis de convergencia}: Verificación teórica vs. práctica
    \item \textbf{Rendimiento comparativo}: Identificación del método más eficiente
    \item \textbf{Recomendaciones específicas}: Sugerencias para optimización
\end{itemize}

\section{Arquitectura del Sistema}

\subsection{Tecnologías Utilizadas}

La aplicación utiliza un stack tecnológico moderno y robusto:

\begin{itemize}
    \item \textbf{Frontend}: React 18 con TypeScript para type safety y mejor experiencia de desarrollo
    \item \textbf{Build Tool}: Vite para compilación rápida y hot reloading optimizado
    \item \textbf{UI Framework}: shadcn/ui con Tailwind CSS para interfaces modernas y responsivas
    \item \textbf{Routing}: TanStack Router para navegación tipo-segura y lazy loading
    \item \textbf{Icons}: Tabler Icons y Lucide React para iconografía consistente
    \item \textbf{Algoritmos}: Implementaciones puras en TypeScript sin dependencias externas
\end{itemize}

\subsection{Estructura Modular}

Cada método numérico (tanto de raíces como iterativo) sigue una arquitectura consistente:

\begin{itemize}
    \item \textbf{Algoritmo Core}: Implementación pura en \texttt{/src/utils/algorithms/}
    \item \textbf{Componente Principal}: Interfaz de usuario en \texttt{/src/features/methods/}
    \item \textbf{Tabla de Resultados}: Componente especializado para visualización tabular
    \item \textbf{Entrada de Datos}: Componentes reutilizables para configuración
    \item \textbf{Ruta}: Configuración de navegación usando file-based routing
\end{itemize}

\subsection{Componentes Compartidos}

\subsubsection{MatrixInput}
Componente unificado para entrada de sistemas lineales que incluye:
\begin{itemize}
    \item Modo texto con sintaxis flexible (espacios o comas)
    \item Validación en tiempo real de matrices hasta $7 \times 7$
    \item Ejemplos precargados ($2 \times 2$, $3 \times 3$, $4 \times 4$)
    \item Verificación de dominancia diagonal automática
    \item Estimación de parámetros óptimos (factor $\omega$ para SOR)
\end{itemize}

\subsubsection{Utilidades de Matriz}
Funciones especializadas en \texttt{/src/utils/matrix-utils.ts}:
\begin{itemize}
    \item Parsing y validación de entrada
    \item Verificación de dominancia diagonal
    \item Cálculo de radio espectral aproximado
    \item Estimación de factor $\omega$ óptimo
    \item Generación de ejemplos educativos
\end{itemize}

\section{Funcionalidades de la Interfaz}

\subsection{Entrada de Datos Unificada}

\subsubsection{Para Métodos de Raíces}
\begin{itemize}
    \item \textbf{Funciones Matemáticas}: Soporte completo para funciones algebraicas usando sintaxis JavaScript
    \item \textbf{Operadores Soportados}: funciones trigonométricas, exponencial, logaritmo, raíz cuadrada
    \item \textbf{Validación en Tiempo Real}: Evaluación inmediata para detectar errores
    \item \textbf{Ejemplos Contextuales}: Placeholders específicos con función por defecto
\end{itemize}

\subsubsection{Para Métodos Iterativos}
\begin{itemize}
    \item \textbf{Matriz y Vector}: Entrada flexible con validación automática
    \item \textbf{Guess Inicial}: Vector de aproximación inicial configurable
    \item \textbf{Parámetros de Convergencia}: Tolerancia y límite de iteraciones
    \item \textbf{Configuración SOR}: Factor $\omega$ automático o personalizado
\end{itemize}

\subsection{Visualización Avanzada de Resultados}

\subsubsection{Métricas Principales}
\begin{itemize}
    \item \textbf{Estado de Convergencia}: Badges visuales con códigos de color
    \item \textbf{Radio Espectral}: Indicador visual de convergencia teórica
    \item \textbf{Eficiencia}: Iteraciones y tiempo de ejecución
    \item \textbf{Precisión}: Error final con notación científica
\end{itemize}

\subsubsection{Tablas Interactivas}
\begin{itemize}
    \item \textbf{Historial Completo}: Todas las iteraciones con formateo numérico
    \item \textbf{Exportación CSV}: Datos completos para análisis externo
    \item \textbf{Truncamiento Inteligente}: Primeras 50 iteraciones para rendimiento
    \item \textbf{Scroll Responsive}: Adaptación a diferentes tamaños de pantalla
\end{itemize}

\section{Sistemas de Comparación}

\subsection{Comparación de Métodos de Raíces}

Análisis automático que ejecuta todos los métodos de búsqueda de raíces con los mismos parámetros:

\subsubsection{Criterios de Evaluación}
\begin{enumerate}
    \item Convergencia exitosa a una raíz válida
    \item Menor número de iteraciones entre métodos convergentes
    \item Tiempo de ejecución optimizado
    \item Precisión del resultado final
\end{enumerate}

\subsection{Comparación de Métodos Iterativos}

Sistema especializado para análisis de sistemas lineales:

\subsubsection{Análisis Teórico vs. Práctico}
\begin{itemize}
    \item Predicción de convergencia basada en radio espectral
    \item Verificación experimental de las predicciones teóricas
    \item Análisis de discrepancias entre teoría y práctica
\end{itemize}

\subsubsection{Identificación del Método Óptimo}
Algoritmo que considera:
\begin{enumerate}
    \item Convergencia exitosa (requisito obligatorio)
    \item Menor número de iteraciones
    \item Tiempo de ejecución más rápido
    \item Radio espectral más favorable
\end{enumerate}

\section{Casos de Uso y Validación}

\subsection{Ejemplos por Defecto}

\subsubsection{Función de Prueba para Raíces}
\[
f(x) = \log(\sin^2(x) + 1) - \frac{1}{2}
\]
Esta función permite evaluación comparativa efectiva debido a su comportamiento no lineal complejo.

\subsubsection{Sistema Lineal de Prueba}
Matriz $3 \times 3$ diagonalmente dominante:
\[
A = \begin{pmatrix}
4 & -1 & 0 \\
-1 & 4 & -1 \\
0 & -1 & 4
\end{pmatrix}, \quad b = \begin{pmatrix}
3 \\ 7 \\ 6
\end{pmatrix}
\]

Esta matriz garantiza convergencia para todos los métodos iterativos y permite demostrar las diferencias de rendimiento.

\subsection{Validación de Resultados}

\subsubsection{Precisión Numérica}
\begin{itemize}
    \item Tolerancias configurables desde $10^{-1}$ hasta $10^{-15}$
    \item Manejo de aritmética de punto flotante
    \item Detección de estancamiento numérico
\end{itemize}

\subsubsection{Casos Límite}
\begin{itemize}
    \item Matrices mal condicionadas
    \item Sistemas con convergencia lenta
    \item Matrices sin dominancia diagonal
\end{itemize}

\section{Manejo de Casos Especiales}

\subsection{Validación Robusta}

\subsubsection{Para Métodos de Raíces}
\begin{itemize}
    \item Sintaxis de funciones matemáticas
    \item Condiciones de convergencia específicas por método
    \item Parámetros numéricos en rangos válidos
\end{itemize}

\subsubsection{Para Métodos Iterativos}
\begin{itemize}
    \item Matrices cuadradas hasta $7 \times 7$
    \item Elementos diagonales no nulos
    \item Coherencia dimensional entre matriz y vectores
    \item Factor $\omega$ en rango válido $(0, 2)$ para SOR
\end{itemize}

\subsection{Manejo de Errores Específicos}

\subsubsection{Algoritmos de Raíces}
\begin{itemize}
    \item División por cero en derivadas
    \item Ausencia de cambio de signo en métodos de intervalo
    \item Convergencia a puntos de inflexión
\end{itemize}

\subsubsection{Métodos Iterativos}
\begin{itemize}
    \item Matrices singulares o mal condicionadas
    \item Divergencia por radio espectral $\geq 1$
    \item Stancamiento por tolerancia demasiado restrictiva
\end{itemize}

\section{Experiencia de Usuario}

\subsection{Diseño de Interfaz Moderna}

\begin{itemize}
    \item \textbf{Navegación Organizada}: Sidebar categorizado por tipo de método
    \item \textbf{Diseño Responsivo}: Adaptación fluida a móviles, tablets y desktop
    \item \textbf{Tema Dual}: Soporte completo para modo oscuro y claro
    \item \textbf{Feedback Visual}: Estados de carga, progreso y resultados
    \item \textbf{Accessibility}: Navegación por teclado y etiquetas semánticas
\end{itemize}

\subsection{Flujo de Trabajo Optimizado}

\begin{itemize}
    \item \textbf{Valores por Defecto}: Ejemplos precargados para prueba inmediata
    \item \textbf{Validación Proactiva}: Verificación en tiempo real
    \item \textbf{Exportación Integrada}: Descarga de resultados en un clic
    \item \textbf{Comparación Opcional}: El usuario decide si ejecutar análisis comparativo
\end{itemize}

\section{Conclusiones y Trabajo Futuro}

\subsection{Logros del Proyecto}

El desarrollo de Senku Solver representa una implementación exitosa y completa de:

\begin{enumerate}
    \item \textbf{Suite Dual}: Métodos de raíces (6 algoritmos) y métodos iterativos (3 algoritmos)
    \item \textbf{Interfaz Unificada}: UX consistente entre diferentes categorías de métodos
    \item \textbf{Análisis Comparativo Dual}: Sistemas especializados para cada tipo de problema
    \item \textbf{Arquitectura Escalable}: Base sólida para futuras extensiones
    \item \textbf{Calidad Académica}: Implementaciones fieles a la teoría numérica
\end{enumerate}

\subsection{Innovaciones Técnicas}

\begin{itemize}
    \item \textbf{Radio Espectral Automático}: Cálculo y predicción de convergencia en tiempo real
    \item \textbf{Optimización SOR}: Estimación automática del factor $\omega$ óptimo
    \item \textbf{Validación Inteligente}: Verificación específica por método y contexto
    \item \textbf{Comparación Automática}: Análisis y recomendaciones sin intervención manual
\end{itemize}

\subsection{Extensiones Planificadas}

\subsubsection{Mejoras Inmediatas}
\begin{itemize}
    \item \textbf{Visualización Gráfica}: Gráficas de convergencia y comportamiento iterativo
    \item \textbf{Métodos Adicionales}: Implementación de variantes especializadas
    \item \textbf{Exportación Avanzada}: Formatos adicionales (PDF, LaTeX)
\end{itemize}

\subsubsection{Desarrollo a Largo Plazo}
\begin{itemize}
    \item \textbf{Optimización WASM}: Migración de algoritmos críticos a WebAssembly
    \item \textbf{API Backend}: Servicios para análisis de matrices grandes
    \item \textbf{Capítulo 3}: Interpolación y aproximación numérica
    \item \textbf{Capítulo 4}: Diferenciación e integración numérica
\end{itemize}

\subsection{Impacto Educativo}

Senku Solver establece un nuevo estándar para herramientas educativas de análisis numérico:

\begin{itemize}
    \item \textbf{Aprendizaje Interactivo}: Experimentación directa con algoritmos
    \item \textbf{Análisis Comparativo}: Comprensión profunda de trade-offs algorítmicos
    \item \textbf{Validación Práctica}: Verificación de conceptos teóricos
    \item \textbf{Accesibilidad Universal}: Disponible en cualquier dispositivo con navegador web
\end{itemize}

La implementación actual establece una base sólida y extensible para el desarrollo continuo de una suite completa de herramientas de análisis numérico, cumpliendo con los más altos estándares académicos mientras proporciona una experiencia de usuario moderna y profesional.
